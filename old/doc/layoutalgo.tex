\documentclass{article}
\begin{document}

%%%%%%%%%%%%%%%%%%%%%%%%%%%%%%%%%%%%%%%%%%%%%%%%%%%%%%%%%%%%%%%%%%%%%%
\subsection{Choosing a layout for the graph}

The procedure |vfreeze| is called near the end and
chooses ``optimal'' vertex positions by minimizing the sum of the
weighted lengths of the arcs.  Since the quantity to be optimized is
quadratic in the vertices
\begin{equation}
  L = \frac{1}{2} \sum_i \sum_{j\in\alpha(i)}
        t_{ij} (\vec v_i - \vec v_j)^2\,,
\end{equation}
where~$\alpha(i)$ denotes the set of vertices connectes with the
vertex~$i$, we just have to solve a system of linear equations.
Adding Lagrange multipliers~$\lambda_{ij}$ for the constraints that
have been specified
\begin{equation}
  \Lambda = \sum_i \sum_{j\in\gamma(i) \atop j>i}
        \vec\lambda_{ij}\cdot(\vec v_i - \vec v_j - \vec d_{ij})\,,
\end{equation}
where~$\gamma(i)$ denotes the set of vertices which have a fixed
distance to the vertex~$i$, the set of linear equations is given by
\begin{eqnarray}
  0 & = & \frac{\partial}{\partial \vec v_i} (L+\Lambda)
           = \sum_{j\in\alpha(i)} t_{ij} (\vec v_i - \vec v_j)
             + \sum_{j\in\gamma(i) \atop j>i} \vec\lambda_{ij}
             - \sum_{j\in\gamma(i) \atop j<i} \vec\lambda_{ji} \\
  0 & = & \vec v_i - \vec v_j - \vec d_{ij}\;\; \mid  j \in \gamma(i)\,.
\end{eqnarray}
The implementation is simple: loop over all vertices in |vlist[]|
and iff the position of the vertex has not been fixed yet, add
another equation.  Note that each arc enters twice (incoming
\emph{and} outgoing) in this procedure.  Thus it is easiest to store
them \emph{both} ways in the |vlist| structure
(cf.~page~\pageref{pg:vconnect})\label{pg:vfreeze}.
Inquiring minds might want to see the result in numbers:
Separate Langrange multipliers for~$x$ and~$y$:

The Lagrange multiplier for a polynomial~$P$
\begin{equation}
  \Lambda_P = \sum_{i=2}^{n_P} \vec\lambda_{P,i}
    \left( \vec v_{i+1} - \vec v_i
            + R(2\pi/n_P) \left( \vec v_{i-1} - \vec v_i \right)
    \right)
\end{equation}
can be reshuffled trivially
\begin{eqnarray}
  \Lambda_P & = &
    \vec\lambda_{n-1} \vec v_{n}
    - \vec\lambda_1 \vec v_2
    + \vec\lambda_2 R(2\pi/n_P) \vec v_1
    - \vec\lambda_n R(2\pi/n_P) \vec v_{n-1} \\
    & & \mbox{} + \sum_{i=2}^{n_P}
        \left( \vec\lambda_{i-1} + \vec\lambda_{i+1} R(2\pi/n_P)
               - \vec\lambda_i (1 + R(2\pi/n_P)) \right) \vec v_i
        \nonumber
\end{eqnarray}
Again, we can ask for positioning and drawing explicitely from \TeX{}
(usually called implicitely by \verb|\end{fmfgraph}|).  This is useful for
subsequent fine tuning.

%%%%%%%%%%%%%%%%%%%%%%%%%%%%%%%%%%%%%%%%%%%%%%%%%%%%%%%%%%%%%%%%%%%%%%
\end{document}

























